\chapter{Estado del arte}
\label{cap:estadoDeLaCuestion}


\section{¿Que es la lectura fácil?}
La lectura fácil es una forma de adaptar textos para una comprensión más sencilla del original. No se trata sólo de un resumen, sino de una simplificación del texto con un lenguaje, vocabulario, términos, oraciones, imágenes descriptivas y formato de forma simple, sencilla y adecuada para aquellas personas con discapacidad intelectual, con dificultad para el lenguaje, con alguna enfermedad y/o trastorno mental, en proceso de aprendizaje, etc. que, según estudios de la Unión Europea, alcanza al 30\% de la población.

\section{Un poco de Historia...}
El movimiento de la Lectura Fácil surgió en Suecia en 1968. En ese año se publicó el primer libro en Lectura Fácil y desde entonces hasta 1994 crearon 330 obras en lectura fácil, entre 15 y 20 nuevas cada año.

Esto se extendió a los países vecinos de Noruega y Finlandia.

En Noruega, por ejemplo, la iniciativa se denomina \textit{“Leser s$\emptyset$ker bok”} (Lector busca libro) y es una alianza de 20 organizaciones, que incluyen editoriales y organizaciones de personas con discapacidad.

En 1988 se creó en Bruselas la organización \textit{Inclusion Europe}, la alianza europea de organizaciones que trabajan por los derechos de las personas con discapacidad, que actualmente agrupa a organizaciones y asociaciones de personas con discapacidad intelectual de 40 países europeos e Israel.

En 1998 se elaboró la guía \textit{«El camino más fácil: Directrices europeas para generar información de fácil lectura destinada a personas con discapacidad intelectual»} y se diseñó un logotipo europeo de lectura fácil, para identificar todos los textos redactados que siguieran sus pautas.

En España fue en 2003 cuando se creó la primera Asociación de Lectura Fácil en Barcelona. Desde entonces, surgen diversas organizaciones e iniciativas en pro de la lectura fácil por toda España.

\section{¿Cómo se identifican los textos de lectura fácil?}
En los textos adaptados a Lectura fácil vienen determinados por dos tipos de logotipos. En la figura \ref{fig:IFLA} es el logo que la Asociación de Lectura Fácil otorga a los textos que se adaptan a las normas de la IFLA. Y la figura  \ref{fig:logoEuropeo}
Logo fomentado por Inclusion Europe.
\begin{figure}[htb]
\centering
	\includegraphics[width=0.5\textwidth]{Imagenes/Logos/indice}
	\caption{Logo LF que cumplen las normas de la IFLA}
	\label{fig:IFLA}
\end{figure} 
\begin{figure}[htb]
	\centering
	\includegraphics[width=0.3\textwidth]{Imagenes/Logos/indice2}
	\caption{Logotipo europeo de LF}
	\label{fig:logoEuropeo}
\end{figure} 
\section{Niveles de adaptación}
Es imposible adaptar un texto para todas las personas que necesiten este tipo de lectura.
 
La IFLA (Federación Internacional de Asociaciones de Bibliotecarios y Bibliotecas) distingue entre los siguientes niveles de adaptación:
\begin{itemize}
	\item Primer nivel, es el más sencillo y simple con muchas imágenes y escaso texto, con una dificultas sintáctica baja.
\item Segundo nivel, es intermedio, menos sencillo que el anterior con un vocabulario y expresiones que son conocidas por todos, fácil de seguir y comprender e imágenes.
\item Tercer nivel, es el más complejo, con textos largos, palabras que no se conocen a menudo, con saltos en el tiempo y muy pocas imágenes. 
 \end{itemize}
\section{Pautas a seguir para la elaboración de lectura fácil}
 El primero documento de como elaborar texto adaptado a lectura fácil fue publicado por la IFLA. Hay otro que fue elaborado por varias organizaciones de Inclusion Europe bajo el título "Información para todos". 
 
 Las pautas que se deben seguir se vincula a la ortografía, gramática, léxico, estilo, diseño, imágenes, formato, etc. 
 
 Algunos ejemplos y recomendaciones son las siguientes:
 \begin{itemize}
 \item Uso de frases simples, cortas y con una estructura habitual.
 \item Uso de imágenes sencillas y pictogramas de apoyo al texto, de manera descriptiva.
 \item Cada frase debe ocupar una línea. Si no es fuese posible deberá ocupar varias líneas.
 \item Evitar oraciones impersonales y pasivas reflejas.
 \item Evitar el subjuntivo o la voz pasiva.
 \item Evitar signos ortográficos poco habituales. (%, &, / ...)
 \item Evitar abreviaturas, acrónimos y siglas
 \item Uso de palabras de uso cotidiano y evitar tecnicismos.
 \item Ideas principales
 \item Redactar en modo directo
 \item No dar conocimientos previos por conocidos.
 \item Evitar diseños cargado.​
\end{itemize}
\section{Movimientos y asociaciones}
\begin{itemize}
	\item Asociación de Lectura Fácil de Barcelona.
	\item Asociación Lectura Fácil Extremadura.
	\item Fundación Ciudadanía (Extremadura)
	\item Dilee Lectura Fácil (Extremadura).
	\item Lectura Fácil Madrid.
	\item Cooperativa Altavoz (Madrid)
	\item Lectura Fácil Euskadi. (Bilbao)
	\item Asociación Aragonesa de Lectura Fácil. (Zaragoza)
	\item Lectura Fácil Castilla y León (Palencia)
	\item Asociación de Lectura Fácil \item “Residencia San Andrés” (Éibar)
	\item Instituto de Lectura Fácil (Sevilla).
\end{itemize}

\section{Proyectos}
TeCuento, aplicación gratuita, para que niños y adultos puedan editar de forma sencilla y divertida sus propios cuentos en lengua de signos española.

Tu Biblio+Fácil busca transformar las bibliotecas públicas en espacios inclusivos, mejorar las habilidades para la lectura (y el gusto por la misma) de las personas con síndrome de Down y desarrollar su autonomía en el uso de espacios públicos.


BraiBook es un dispositivo de lectura capaz de convertir cualquier documento de texto, en formato electrónico y en cualquier idioma, al código braille.


Videolibros enSeñas, es la primera biblioteca virtual, libre y gratuita en lengua de señas y con voz en español, que ideamos como solución innovadora para que las niñas, niños y adolescentes sordos accedan a la literatura infantil. 


Dyseggxia es un juego para teléfonos móviles que ayuda a los niños con dislexia a superar sus problemas de lectura y escritura en castellano a través de divertidos juegos. 


Voicebook es un lector de libros pionero que te permite leer textos digitales como E-pub y PDF para que puedas leer novelas, cuentos, artículos… 


La Mesita es una aplicación de descarga gratuita para dispositivos táctiles, recomendado para tablets. 


Sanapalabras El 10\% de la población tiene dislexia, algunos padres no saben que esto afecta el rendimiento de los niños en la escuela. 


Yo también leo es una aplicación diseñada especialmente para que niños con síndrome de Down y otros tipos de diversidad funcional cognitiva aprendan a leer con una metodología adaptada a sus capacidades.
%En el estado de la cuestión es donde aparecen gran parte de las referencias bibliográficas del trabajo. Una de las formas más cómodas de gestionar la bibliografía en {\LaTeX} es utilizando \textbf{bibtex}. Las entradas bibliográficas deben estar en un fichero con extensión \textit{.bib} (con esta plantilla se proporcionan 3, dos de los cuales están vacíos). Cada entrada bibliográfica tiene una clave que permite referenciarla desde cualquier parte del texto con los siguiente comandos:

%\begin{itemize}
%\item Referencia bibliografica con cite: \cite{ldesc2e}
%\item Referencia bibliográfica con citep: \citep{notsoshort}
%\item Referencia bibliográfica con citet: \citet{latexAPrimer}
%\end{itemize}

%Es posible citar más de una fuente, como por ejemplo %\citep{latexCompanion,LaTeXLamport,texKnuth}

%Después, latex se ocupa de rellenar la sección de bibliografía con las entradas \textbf{que hayan sido citadas} (es decir, no con todas las entradas que hay en el .bib, sino sólo con aquellas que se hayan citado en alguna parte del texto).

%Bibtex es un programa separado de latex, pdflatex o cualquier otra cosa que se use para compilar los .tex, de manera que para que se rellene correctamente la sección de bibliografía es necesario compilar primero el trabajo (a veces es necesario compilarlo dos veces), compilar después con bibtex, y volver a compilar otra vez el trabajo (de nuevo, puede ser necesario compilarlo dos veces). 
