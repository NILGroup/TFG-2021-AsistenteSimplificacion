%---------------------------------------------------------------------
%
%                         TeXiS_bib.tex
%
%---------------------------------------------------------------------
%
% TeXiS_bib.tex
% Copyright 2009 Marco Antonio Gomez-Martin, Pedro Pablo Gomez-Martin
%
% This file belongs to TeXiS, a LaTeX template for writting
% Thesis and other documents. The complete last TeXiS package can
% be obtained from http://gaia.fdi.ucm.es/projects/texis/
%
% This work may be distributed and/or modified under the
% conditions of the LaTeX Project Public License, either version 1.3
% of this license or (at your option) any later version.
% The latest version of this license is in
%   http://www.latex-project.org/lppl.txt
% and version 1.3 or later is part of all distributions of LaTeX
% version 2005/12/01 or later.
%
% This work has the LPPL maintenance status `maintained'.
% 
% The Current Maintainers of this work are Marco Antonio Gomez-Martin
% and Pedro Pablo Gomez-Martin
%
%---------------------------------------------------------------------
%
% Fichero que contiene la generación de la bibliografía. En principio,
% no haría falta tenerlo en  un fichero separado, pero como permitimos
% añadir una frase célebre antes  de la primera cita, la configuración
% ya  no es  trivial.  Para  "ocultar" la  fontanería  de LaTeX,  está
% separada  la  configuración  de  los  parámetros  de  la  generación
% concreta de la bibliografía. La configuración está en el "directorio
% del usuario" (Cascaras), mientras  que la generación se encuentra en
% el directorio TeXiS (este fichero).
%
%---------------------------------------------------------------------

%%%
% Gestión de la configuración
%%%

% Ficheros .bib
\def\ficherosBibliografia{}
\newcommand{\setBibFiles}[1]{
\def\ficherosBibliografia{#1}
}

% Frase célebre
\def\citaBibliografia{}
\newcommand{\setCitaBibliografia}[1]{
\def\citaBibliografia{#1}
}

%%%
% Configuración terminada
%%%

%%%
%% COMANDO PARA CREAR LA BIBLIOGRAFÍA.
%% CONTIENE TODO EL CÓDIGO LaTeX
%%%
\newcommand{\makeBib}{

%
% Queremos que  tras el título del  capítulo ("Bibliografía") aparezca
% una frase célebre,  igual que en el resto  de capítulos. El problema
% es que  aquí no ponemos  nosotros a mano  el \chapter{Bibliografía},
% sino que lo mete él automáticamente.
%
% Afortunadamente,  la gente  de  bibtex  hace las  cosas  bien ;-)  y
% después  de insertar  el título  de  la sección  ejecuta un  comando
% denominado  \bibpreamble  que por  defecto  no  hace  nada. Pero  si
% sobreescribimos  ese comando,  podremos  ejecutar código  arbitrario
% justo  después de la  inserción del  título, y  antes de  la primera
% referencia. Por  tanto, lo que hacemos es  sobreescribir ese comando
% (normalmente  se conocen  como "hooks")  para añadir  la  cita justo
% después del título.
%
% Desgraciadamente,  dependiendo  de la  versión  de  Natbib, hay  que
% definir  o redefinir  el comando  (es decir,  utilizar  newcommand o
% renewcommand)... como eso es un  lío, utilizamos let y def, pues def
% no falla si ya estaba definido.

\let\oldbibpreamble\bibpreamble

\def\bibpreamble{%
\oldbibpreamble
% Añadimos a la tabla de contenidos la bibliografía. Si no lo hacemos
% aquí, sale mal o el número de página (grave) o el enlace en el PDF
% que te lleva a un sitio cercano (no tan grave)
%\ifx\generatoc\undefined
%\else
%\addcontentsline{toc}{chapter}{Bibliografía}
\addcontentsline{toc}{chapter}{\bibname}
%\fi
% Añadimos también una etiqueta, para poder referenciar el número
% de página en el que comienza
\label{bibliografia}
% Frase célebre configurada por el usuario
\citaBibliografia
}
% Fin definición "bibpreamble"


%
% Cambiamos  el estilo  de la  cabecera.  Hay que  hacerlo porque  por
% defecto el paquete que  estamos usando (fancyhdr) pone en mayúsculas
% el título completo del capítulo.  Con los capítulos normales esto se
% pudo evitar  en el  preámbulo redefiniendo el  comando \chaptermark,
% pero con la bibliografía no se puede hacer. Se define la cabera para
% que aparezca la palabra "Bibliografía" en ambas páginas.
%
%
%\cabeceraEspecial{Bibliografía}


% Creamos la  bibliografía. Lo hacemos  dentro de un bloque  (entre la
% pareja \begingroup  ... \endgroup) porque  dentro vamos a  anular la
% semántica que da babel  a la tilde de la eñe (que  hace que un ~N se
% convierta  automáticamente en una  Ñ). Esto  es debido  a que  en el
% bibtex  aparecerán  ~ para  separar  iniciales  de  los nombres  con
% espacios  no separables  en varias  lineas, y  aquellos  nombres que
% tengan  una N  como  inicial serían  puestos  como Ñ.  Al anular  la
% semántica     al    ~    que     da    babel,     deshacemos    este
% comportamiento. Naturalmente,  para que esto  no tenga repercusiones
% negativas,  en  ningún  .bib  deberíamos  utilizar ~N  (o  ~n)  para
% representar una ñ ... tendremos  que utilizar o una Ñ/ñ directamente
% (no  aconsejable porque asume  que hemos  usado inputenc)  o, mejor,
% usamos la  versión larga  \~n o  \~N que no  falla nunca.   Para más
% información,  consulta  el  TeXiS_pream.tex  en el  punto  donde  se
% incluye natbib y babel.

\begingroup
\spanishdeactivate{~}
% Según la documentación de los ficheros .bst que se puede encontrar en
% http://osl.ugr.es/CTAN/info/bibtex/tamethebeast/ttb_en.pdf
% hacer ficheros de estilo multilenguaje es complicado
% Se opta por hacer dos estilos separados: uno para español y otro para
% inglés. Descomentar el que se quiera utilizar

% Descomentar para bibliografía en español
\bibliographystyle{TeXiS/TeXiS}
% Descomentar para bibliografía en inglés
%\bibliographystyle{TeXiS/TeXiS_en}

\bibliography{\ficherosBibliografia}
\endgroup

} %\newcommand{\makeBib}

% Variable local para emacs, para  que encuentre el fichero maestro de
% compilación y funcionen mejor algunas teclas rápidas de AucTeX

%%%
%%% Local Variables:
%%% mode: latex
%%% TeX-master: "../Tesis.tex"
%%% End:
