\chapter*{Resumen}

Debemos tener en cuenta que, en nuestra sociedad, hay cerca de un 30\% de personas con dificultades de comprensión lectora y de aprendizaje (dislexia, discapacidad intelectual, personas mayores, aquellas que estén iniciándose en el idioma...), convirtiéndose así la Lectura Fácil (LF) en un aliado esencial que permita la accesibilidad de este sector a la información y a la cultura. 

 \setlength{\parskip}{10pt}

En muchas ocasiones aquella persona encargada de transformar textos a LF debe realizar manualmente la tarea de adaptación de textos, lo que supone mucho esfuerzo en detrimento de la eficiencia que podría aumentar si se dispusiera de una aplicación o herramienta que convierta este trabajo manual en otro más rápido y eficaz.

 \setlength{\parskip}{10pt}

Surge así la necesidad de abordar este problema en este TFG llamado ``Asistente web interactivo para la simplificación de textos'' diseñando e implementando un asistente web que tiene la finalidad de ayudar al usuario editor de adaptar textos  bien sean descriptivos, narrativos, periodísticos, etc... a Lectura Fácil de una manera interactiva. 

 \setlength{\parskip}{10pt}

El editor tendrá a su disposición diversas funcionalidades para hacer posible la adaptación, llevando a cabo simplificaciones léxicas en el texto mediante identificación de palabras que puedan conllevar más dificultad para el lector y reemplazándolas por otras más sencillas. Además, permitirá intercambios sintácticos en las frases, supresión de palabras o adición de información en el texto para una mejor comprensión del mismo. Finalmente, el editor obtendrá un borrador del texto final y visualizar el resultado del texto una vez haya sido adaptado, pudiendo hacer sobre él modificaciones adicionales.

 \setlength{\parskip}{10pt}

Con esta aplicación pretendemos favorer a la comprensión y aprendizaje por parte del lector acercándole a una igualdad social y cultural ya que le permite captar de una manera más accesible las ideas que el texto quiere trasmitir.



\section*{Palabras clave}
   
\begin{itemize}
   	\item Acceso a la información.
   	\item Lectura Fácil.
   	\item Simplificación de frases de texto.
   	\item API REST.
   	\item Librería spaCy.
\end{itemize}
   



