\chapter*{Resumen}

Debemos tener en cuenta que, en nuestra sociedad, hay cerca de un 30\% de personas con dificultades de comprensión lectora y de aprendizaje (dislexia, discapacidad intelectual, personas mayores, aquellas que estén iniciándose en el idioma...), convirtiéndose así la Lectura Fácil (LF) en un aliado esencial que permite la accesibilidad de este sector de la población a la información y a la cultura. 

 \setlength{\parskip}{10pt}

En muchas ocasiones aquella persona encargada de transformar textos a LF debe realizar manualmente la tarea de adaptación de textos, lo que supone un gran esfuerzo, en detrimento de la eficiencia que podría aumentar si se dispusiera de una aplicación o herramienta que convierta este trabajo manual en otro más rápido y eficaz.

 \setlength{\parskip}{10pt}

Surge así la necesidad de abordar el problema en este TFG llamado ``Asistente web interactivo para la simplificación de textos'', diseñando e implementando un asistente web que tiene la finalidad de ayudar al usuario editor a adaptar textos a LF bien sean descriptivos, narrativos, periodísticos, etc... de una manera interactiva. 

 \setlength{\parskip}{10pt}

El usuario editor tendrá a su disposición diversas funcionalidades para hacer posible la adaptación, llevando a cabo simplificaciones léxicas en el texto mediante identificación de palabras que puedan conllevar más dificultad para el lector y reemplazándolas por otras más sencillas. Además, permitirá intercambios sintácticos en las frases, supresión de palabras o adición de información en el texto para una mejor comprensión del mismo. Finalmente, el usuario editor obtendrá un borrador del texto final y podrá visualizar el resultado del texto una vez haya sido adaptado, pudiendo hacer sobre él modificaciones adicionales. En todo momento, el encargado de editar texto, tendrá el control absoluto sobre la adaptación, pero apoyado por el asistente para realizar más rápido ciertas simplificaciones.

 \setlength{\parskip}{10pt}

Con esta aplicación pretendemos favorecer la comprensión y el aprendizaje por parte del lector con dificultades lectoras acercándole a una igualdad social y cultural, ya que le permite captar de una manera más accesible las ideas que un texto quiera trasmitir.



\section*{Palabras clave}
   
\begin{itemize}
   	\item Procesamiento Lenguaje Natural.
   	\item Lectura Fácil.
   	\item Simplificación de texto.
   	\item API REST.
   	\item NIL-WS API.
\end{itemize}
   



