\chapter*{Resumen}

Debemos tener en cuenta que en nuestra sociedad, hay cerca de un 30\% de personas con dificultades de comprensión lectora y de aprendizaje (dislexia, discapacidad intelectual, personas mayores…). 


Surge así la necesidad de abordar este problema en este TFG llamado ‘Asistente web interactivo para la simplificación de textos’ diseñando e implementando un asistente web que proporciona ayuda a la persona encargada de editar textos, bien sean descriptivos, narrativos, periodísticos.,etc.,  para adaptarlos a Lectura Fácil en favor de esta parte de la sociedad. 


El editor tendrá a su disposición diversas funcionalidades para hacer posible la adaptación, llevando a cabo simplificaciones léxicas en el texto mediante identificación de palabras que puedan conllevar más dificultad para el lector y reemplazándolas por otras más sencillas.


 Además, permitirá intercambios sintácticos en las frases, supresión de palabras o adición de información en el texto para una mejor comprensión del mismo. Finalmente, el editor obtendrá un borrador del texto final y visualizar el resultado del texto una vez haya sido adaptado.


Con esta aplicación favorecemos a la comprensión y aprendizaje por parte del lector así como a una igualdad dentro de la sociedad ya que nos permite captar de una manera más accesible las ideas que el texto quiere trasmitir.



\section*{Palabras clave}
   
\noindent Máximo 10 palabras clave separadas por comas

   


