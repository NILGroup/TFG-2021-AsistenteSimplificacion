\chapter*{Abstract}

We must take into account that, in our society, there are about 30\% of people with reading comprehension and learning difficulties (dyslexia, intellectual disability, elderly people, those who are starting to learn the language...), making Easy Reading (ER) an essential ally that allows the accessibility of this sector of the population to information and culture. 

On many occasions, the person in charge of transforming texts into ER must manually perform the task of adapting texts, which is a great effort, to the detriment of the efficiency that could be increased if an application or tool were available to convert this manual work into a faster and more efficient one.

Thus arises the need to address the problem in this FDP called ``Interactive web assistant for text simplification'', designing and implementing a web assistant that aims to help the user editor to adapt texts to ER whether descriptive, narrative, journalistic, etc... in an interactive way. 

The user editor will have at his disposal several functionalities to make the adaptation possible, carrying out lexical simplifications in the text by identifying words that may be more difficult for the reader and replacing them with simpler ones. In addition, it will allow syntactic exchanges in the sentences, deletion of words or addition of information in the text for a better understanding of the text. Finally, the user editor will obtain a draft of the final text and will be able to visualize the result of the text once it has been adapted, being able to make additional modifications. At all times, the text editor will have full control over the adaptation, but supported by the wizard to make certain simplifications faster.

With this application we intend to promote understanding and learning by readers with reading difficulties, bringing them closer to a social and cultural equality, as it allows them to grasp in a more accessible way the ideas that a text wants to convey.


\section*{Keywords}

\noindent Natural Language Processing, Easy Reading, text simplification, REST API, NIL-WS-API.
