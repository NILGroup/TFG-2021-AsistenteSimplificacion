\chapter{Herramientas}
\label{cap:herramientas}

En este capítulo hablaremos sobre las distintas tecnologías utilizadas para el desarrollo del trabajo. Expondremos los motivos por los cuales hemos decidimos usar unas tecnologías frente a otras y hablaremos sobre los
orígenes de las mismas.


\section{Flask}
Flask es un framework minimalista escrito en Python que permite crear aplicaciones web rápidamente y con un mínimo número de líneas de código. Está basado en la especificación WSGI de Werkzeug y el motor de templates Jinja2 y tiene una licencia BSD.
\section{Spacy}
Traducción del inglés-spaCy es una biblioteca de software de código abierto para el procesamiento avanzado del lenguaje natural, escrito en los lenguajes de programación Python y Cython.
\section{Postman}
Postman es una herramienta que se utiliza, sobre todo, para el testing de API REST, aunque también admite otras funcionalidades que se salen de lo que engloba el testing de este tipo de sistemas.

Gracias a esta herramienta, además de testear, consumir y depurar API REST, podremos monitorizarlas, escribir pruebas automatizadas para ellas, documentarlas, mockearlas, simularlas, etc.