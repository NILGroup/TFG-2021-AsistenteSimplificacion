\chapter{Introduction}
\label{cap:introduction}

\section{Motivation}
Nowadays, human beings have a multitude of ways to promote learning. One of them is reading. In the world in which we live we can exercise it practically through various media, either through books, social networks, television or the press. It is a right that everyone should have within their reach. That way, a reader can relate words, symbols, images or numbers within his or her mind, and thus learn.

 \setlength{\parskip}{10pt}

There are circumstances that make this reading process not so trivial, but a ``help'' is needed to make it more accessible to understand the information being transmitted. This is the case of those people who have some kind of cognitive disability, advanced age or are unfamiliar with the language for whatever reason, causing a barrier between them and reading.

 \setlength{\parskip}{10pt}

Overcoming this barrier is the main objective pursued by Easy Reading (ER), favoring accessibility to these people by means of adapted texts so that a reading that they perceive as complex to understand becomes a text that transmits, in a simpler way, the same idea, thus facilitating communication between the reading and the reader.

 \setlength{\parskip}{10pt}

The manual adaptation of texts or documents to Easy Reading is very costly and time-consuming. Taking into account that the appearance of new information (news, blogs, social networks...) is constantly growing, it would be necessary to maintain the same pace of adaptation to ER. However, a difficult task to manually elaborate a text that meets the characteristics of ER. 

 \setlength{\parskip}{10pt}

Easy Reading is not only about detecting difficult words or making shorter sentences, but sometimes it is necessary, for example, to restructure the syntax of a text completely or to synthesize or expand the original information. For an adaptation it is necessary to keep in mind the addressee or group of addressees, with the information, needs, interests or previous knowledge they may have about a certain topic. This makes the human factor very important and gives strength to the ER with very well done, thought and worked adaptations.

 \setlength{\parskip}{10pt}

We live in a world surrounded by technological advances, having at our disposal a large number of devices or tools. There are more and more initiatives that make use of them to overcome these barriers and make reading possible for everyone, adapting them to accessible formats. 

 \setlength{\parskip}{10pt}

Thus, the idea of developing an application that combines functionalities that meet the characteristics and actions of the ER arises, providing the user-editor with additional help, minimizing his effort when he has to adapt a text to ER.

\section{Objectives}
This application consists of a tool that puts at the service of the text editor, besides being able to make manual adjustments at all times, a series of features and actions in a visual and interactive way that helps to convert original texts to others with a clearer and more concise language in a faster way.  

 \setlength{\parskip}{10pt}

The main objective is to help people by facilitating the adaptation of texts to Easy Reading with the use of an interactive application that allows them to make syntactic and lexical transformations through the use of natural language processing techniques. 



\section{Document structure}


We have followed a series of steps for the development of this report, which is divided into the following chapters:

 \setlength{\parskip}{10pt}
 \begin{itemize}
\item {\textbf{Chapter 1}} (Introduction, motivation and structure of the document): in this chapter, also translated into English, we make a small introduction to understand the problem of reading difficulties from which this dissertation arises, the motivation and structure with explanations about what we are going to expose in each of the chapters.

 \setlength{\parskip}{10pt}
 
\item {\textbf{Chapter 2}} (State of the art): in this chapter, we explain what Easy Reading is, how it arises, to whom it is addressed, its identification, guidelines, levels and tasks in an adaptation. We will also talk about associations, applications and materials adapted to Easy Reading. 

 \setlength{\parskip}{10pt}

\item {\textbf{Chapter 3}} (Tools and technologies used): in this chapter we will talk about the technologies we have used for the development of the application.

 \setlength{\parskip}{10pt}

\item {\textbf{Chapter 4}} (Interactive web assistant for the simplification of texts to Easy Reading): in this chapter we will describe the necessary requirements that the web application must fulfill. In addition, we will go through the different interfaces of the application and the details of its use. 

 \setlength{\parskip}{10pt}

\item {\textbf{Chapter 5}} (Implementation): in this chapter we will detail the architecture on which our wizard is based, as well as how the different external web services and libraries of both the server side and the application have been implemented. 

 \setlength{\parskip}{10pt}

\item{\textbf{Chapter 6}} (Conclusions and future work): in this chapter, also in English, we will make some final evaluations about the wizard and give some ideas about some future developments that could be covered.

 \setlength{\parskip}{10pt}

\item{\textbf{Chapter 7}} (Individual work): in this chapter we show the contributions made by each of the members during the project.

\end{itemize}







