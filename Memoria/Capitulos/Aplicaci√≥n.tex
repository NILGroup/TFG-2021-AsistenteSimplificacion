\chapter{Aplicación}
\label{cap:aplicacion}

En este capítulo hablaremos sobre el proceso de diseño de la aplicación, las dificultades que nos hemos ido encontrando en su desarrollo, los cambios que ha ido sufriendo la aplicación y las mejoras hasta llegar a la versión final de la aplicación.

Vamos a dividir en dos secciones, frontend y backend.

\section{Frontend}

El frontend es la parte de un sitio web que interactúa con los usuarios, es decir, está en la parte del cliente. 

Hay varios lenguajes que se usan para el desarrollo del fronted, nosotros vamos a usar tres: HTML, CSS y javaScript, que a continuación haremos una breve descripción de cada uno:
 
\begin{itemize}
	\item HTML son las siglas en inglés \textit{HyperText Markup Lenguage} que significa lenguaje de marcado de hiperTexto, una herramienta de elaboración de páginas web. Se trata de un conjunto de etiquetas que sirven para definir el texto y otros elementos que compondrán la página como imágenes, listas, vídeos, etc. Los ficheros HTML, tienen una extensión .html.
	
	\item CSS son las siglas en inglés \textit{Cascading Style Sheets} que significa hoja de estilos en cascada, que sirve para aplicar estilos (colores, tamaños, alineación...) a un fichero HTML (páginas web). Los ficheros CSS, tienen una extensión .css.
	
		\item JavaScript es un lenguaje de programación interpretado, no es necesario compilarlo para ejecutardo, que se usa para crear páginas web dinámicas. Incorpora efectos, acciones e a ejecutar, animaciones, cambio de estilo, etc., que permite la interactividad con el usuario. Se ejecuta siempre en la parte del cliente. Los ficheros JavaScript, tienen una extensión .js.
	
\end{itemize}

\section{Backend}

El backend se encarga de toda la lógica necesarias para que la web funcione correctamente, siendo estos invisibles. Se trata de una serie de procesos o funciones no podemos ver.