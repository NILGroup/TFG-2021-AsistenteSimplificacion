\chapter{Conclusiones y Trabajo Futuro}
\label{cap:conclusiones}

En este apartado queremos poner en valor a todas aquellas personas encargadas de facilitar la compresión y el aprendizaje de un determinado texto a esa parte de la sociedad que, desgraciadamente, no tienen la suficiente capacidad psíquica y/o física para comprender lo que el texto quiere transmitir.

Por nuestra parte, hemos querido contribuir con nuestro granito de arena para que este trabajo se haga lo menos costoso posible para el editor, ya que es importante que cualquier persona, a pesar de sus discapacidades, tenga acceso a la cultura y a la información de una manera u otra.

Previa a la implementación de este asistente, tuvimos que realizar una labor de investigación para ver cual era el proceso que seguían estos editores para convertir textos a Lectura Fácil, concluyendo que, a día de hoy, necesitaban todavía de procesos manuales para llevar a cabo estas adaptaciones.

De esta manera este asistente se implementó teniendo presente el objetivo de convertir frases complejas a otras más simples, mediante simplificaciones léxicas y sintácticas para que le resulte al lector más sencillo de entender, proporcionando al editor varias funcionalidades que permitan estos objetivos. Con la ayuda del árbol de dependencias el usuario podrá llevar a cabo estas simplificaciones de una manera interactiva, simplemente seleccionando palabras y acción, pudiendo visualizar en todo momento el resultado provisional de estas adaptaciones.


Con respecto al diseño, pretendemos que la interfaz implementada sea lo más sencilla e intuitiva posible para hacer lo más ameno posible la tarea al editor.

Este asistente presenta mejoras técnicas que pueden ser implementadas en próximos desarrollos, explicamos algunas de ellas:

Referente a la funcionalidad, sería interesante abordar los siguientes puntos:

\begin{itemize}
	\item	Al reemplazar un sinónimo, éste esté conjugado y concuerde con el resto de la oración automáticamente sin necesidad de que el editor lo adapte a las necesidades del contexto de la frase.
		\item	Implementar diferentes niveles de complejidad de la adaptación ya que no todos los lectores tendrán las mismas necesidades.
	\item Importación de imágenes para explicar algún concepto que pueda ser algo más complejo o que el lector desconozca, y de esta manera resulte más claro el texto.
	\item	Trabajar en los servicios que ofrece la API de NIL WS para que haga uso de las reglas de acentuación del castellano, ya que todas las palabras vienen dadas sin signos de puntuación.
	\item	Extender estos servicios a otros idiomas de manera que el asistente sea capaz de reconocer diferentes tipos de lenguajes.
\end{itemize}	
Con respecto a la portabilidad de la aplicación web, proponemos la siguiente mejora:
\begin{itemize}
	\item	Sería interesante desarrollar esta aplicación para dispositivos Android e IOS para una mayor comodidad del editor, pudiendo, por ejemplo, realizar estas adaptaciones desde una Tablet, aprovechando el uso táctil a la hora de elegir palabras o funcionalidades.

\end{itemize}

