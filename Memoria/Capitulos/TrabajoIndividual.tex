\chapter{Trabajo individual}
\label{cap:trabajoIndividual}

\section{Estefanía Ortega Ávila}

En primer lugar, mi labor en este TFG se centró en una investigación de las tareas y los pasos a seguir que se llevan a cabo para adaptar un texto a Lectura Fácil. Para ello, consulté diversas webs y vídeos que me proporcionaran dicha información. De esta manera, entendí a qué público iba dirigida este tipo de lectura y qué técnicas eran las más utilizadas. 
Además, tanto mi compañero como yo, identificamos diferentes herramientas que podrían ser útiles para transformar textos, probando algunas de ellas, haciéndonos una idea de qué medios disponen los editores para realizar su labor.

Una vez realizada la investigación, mi compañero y yo comenzamos a estudiar las tecnologías que iban a hacer posible la implementación de nuestro asistente web. Optamos por hacer uso de servicios de API REST, así como de la librería spaCy para el PLN.

Por otro lado, y esta es la que yo abordé, tenemos la implementación de la aplicación, es decir, la interfaz visual con el que el usuario interactuará y hará uso de las diferentes funcionalidades. Para su desarrollo, utilicé JavaScript para que los resultados de la llamada al servidor mediante una promesa ``fetch'' a un endpoint me devolviera un objeto JSON que, al operar sobre él, plasme cierta información en la interfaz;  HTML, para el desarrollo de la interfaz visual, y CSS para dar estilo a ésta (fuentes, tamaños, colores…).

La parte que más dificultad tuve fue la de dibujar el árbol de dependencias con el aspecto de árbol genealógico y el poder procesar el objeto JSON que recibo del servidor para representar cada nodo con sus correspondientes dependencias. La solución por la que opté fue la de implementar un algoritmo BFS para poder recorrer el objeto, obteniendo la estructura de árbol deseada.

Otra parte de especial dificultad fue la relacionada con la funcionalidad ``Intercambiar'' en el caso de que se tuviera que modificar dos palabras independientes entre sí y, a su vez, las dependientes de éstas, reflejándose estos cambios en el borrador del texto final. Para solventarlo, hice uso de arrays auxiliares que me permitieron obtener el resultado que quería en el borrador para poder cambiar el texto dinámicamente.
Para el diseño de la interfaz del asistente, tomé en cuenta las recomendaciones de mis tutoras en las reuniones periódicas.

Además de la implementación, me encargué de algunas de las partes que consta la memoria. En el capítulo 1, redacté los puntos relacionados con la motivación y los objetivos que persigue este asistente. El capítulo 2, los puntos 2.3 y 2.4 relacionados con proyectos, programas y aplicaciones en LF. Con respecto al capítulo 4, fui la persona encargada de desarrollarlo en su totalidad hablando sobre la aplicación y sus diferentes funcionalidades y vistas, mientras que del capítulo 5 fui la autora del punto 5.1 y 5.3, relacionados con la arquitectura y con la implementación del asistente, respectivamente. Finalmente, el capítulo 6, lo elaboré junto con la ayuda de mi compañero.

Por supuesto, se fue realizando modificaciones en aquellas partes donde nuestras tutoras les parecía que debíamos mejorar.

\section{Javier Sesé García}

Mi contribución al proyecto comenzó investigando a quién iba dirigida la Lectura Fácil y las formas que se utilizaban para adaptar textos. 

Posteriormente, nos centramos en el diseño de la arquitectura de la aplicación, la cual es una arquitectura REST basada en endpoints. Elegimos este diseño ya que nos pareció que es fácilmente ampliable en el caso de que se quisiera realizar un servicio nuevo.

Implementé esta arquitectura con Flask, del cual he adquirido los conocimientos para ser capaz de desplegar el servidor en él. He elegido Flask ya que, aunque tenia más experiencia con Django, me ha parecido un entorno ligero y fácil de usar y una oportunidad de ampliar mis conocimientos.

En lo que concierne a las herramientas, investigué y aprendí a usar la librería spaCy, la cual otorga la principal funcionalidad de análisis gramatical del asistente. 


Además, implementé las llamadas a los diferentes servicios de NIL-WS-API desde la parte del servidor, escrito con lenguaje Python.

Desconocía como se debían de hacer las llamadas a los endpoints del servidor, lo que me supuso un esfuerzo hasta que aprendí a realizar las promesas mediante la operación ``fetch''. Además, implementé parte de la funcionalidad de las propias promesas en JavaScript junto con mi compañera.

Una vez terminada la implementación me encargué de alojar el servidor en el contenedor que nos han otorgado (\url{https://holstein.fdi.ucm.es/tfg/2021/simpli/}). Para ello, tuve que aprender a como usarlo, así como de saber lanzar el servidor de Flask en un docker, lo que ha obligado a realizar una modificación en las rutas de los archivos .js y .css locales.

En cuanto a la memoria además del resumen, redacté, en lo que concierne al capítulo 1, la sección de ``Estructura del documento''. En el capítulo 2  me encargué de los puntos 2.1 y 2.2. El capítulo 3 relacionados con las herramientas lo elaboré al completo. El capítulo 5 escribí las 3 primeras secciones ligadas a la parte del servidor de la aplicación y la librería spaCy, salvo el punto 5.1.1. Junto con la ayuda de mi compañera, redactamos el capítulo 6 para explicar las conclusiones y posibles mejoras.