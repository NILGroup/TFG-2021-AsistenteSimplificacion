\chapter{Introducción}
\label{cap:introduccion}

\chapterquote{La lectura no da al hombre sabiduría; le da conocimientos}{William Somerset Maugham}

\section{Motivación}

La lectura es comprender cualquier tipo de información de manera visual o de forma escrita. El poder leer como por ejemplo un libro no es una obligación, es un derecho que cualquier persona debe tener a su alcance. De eso modo, el lector, debe trasladar palabras, símbolos, imágenes o números dentro de la mente, y así aprender. Hay personas con deterioros funcionales o que por algún otro motivo no pueden permitírselo. 

Hoy en día vivimos en un mundo rodeado de la tecnología, teniendo a nuestra disposición dispositivos electrónicos haciéndonos la vida más fácil e incluso más rápida. Cada vez hay más iniciativas que hacen uso de ellas para poder superar esas barreras y hacer posible la lectura a todo el mundo, adaptándolas a formatos accesibles. 

Puesto que el contenido en la actualidad crece a un ritmo vertiginoso, nos surge la idea del desarrollo de una aplicación, para ayudar a las personas, que hacen posible esas adaptaciones y dándoles apoyo en su día a día.


\section{Objetivos}
Esta aplicación consiste en una herramienta que ponga al servicio del adaptador una serie de funcionalidades de forma visual que ayude a convertir textos originales a otros con un lenguaje más claro y conciso.


