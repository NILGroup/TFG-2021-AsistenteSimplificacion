\chapter{Introducción}
\label{cap:introduccion}

\chapterquote{La lectura no da al hombre sabiduría; le da conocimientos}{William Somerset Maugham}

\section{Motivación}
Hoy en día, el ser humano tiene multitud de formas de favorecer su aprendizaje. Una de ellas es la lectura. El mundo en el que vivimos podemos ejercerla prácticamente a través de diversos medios, bien sea mediante redes sociales, televisión o prensa. Es un derecho que cualquier persona debe tener a su alcance. De ese modo el lector, puede relacionar palabras, símbolos, imágenes o números dentro de la mente, y así aprender.

 \setlength{\parskip}{10pt}

 Se dan circunstancias que hacen que este proceso de lectura no sea tan trivial (por ejemplo en aquellas personas que sufren deterioros cognitivos), sino que se necesita una ''ayuda'' para que le sea más accesible comprender la información que se les transmite. Este es el principal objetivo que persigue la Lectura Fácil (LF), dar accesibilidad a este grupo de la sociedad por medio de textos adaptados de manera que una lectura que perciban como compleja de comprender, se conviertan en textos que transmitan, de una manera más simple, la misma idea facilitando así la comunicación entre la lectura y el lector.
 
 \setlength{\parskip}{10pt}
 
La LF no solo es detectar las palabras difíciles o hacer frases más cortas, sino que en ocasiones es necesario, por ejemplo, reestructurar sintáctica de un texto completamente o realizar una síntesis de la información original o ampliarla. Para una adaptación hay que tener en mente al destinatario o grupo de destinatarios, con la información, necesidades, intereses o conocimientos previos que puedan tener sobre un determinado tema. Esto hace que el factor humano sea muy importante y dé fuerza a la LF con adaptaciones muy bien hechas, pensadas y trabajadas. Actualmente no existen programas informáticos que desempeñen esto. Sí lo hacen algunos que determinan el nivel de legibilidad y comprensión de un texto. Esto sería un reto para el futuro.  

\setlength{\parskip}{10pt}
Vivimos en un mundo rodeado de avances tecnológicos, teniendo a nuestra disposición un gran número de dispositivos o herramientas. Cada vez hay más iniciativas que hacen uso de ellas para poder superar esas barreras y hacer posible la lectura a todo el mundo, adaptándolas a formatos accesibles. 

 \setlength{\parskip}{10pt}
 
Puesto que el contenido en la actualidad crece a un ritmo vertiginoso, nos surge la idea del desarrollo de una aplicación, para ayudar a las personas, que hacen posible esas adaptaciones y dándoles apoyo en su día a día, aunque siempre tendremos en cuenta el factor humano para llevar a cabo esta labor.


\section{Objetivos}
Esta aplicación consiste en una herramienta que ponga al servicio del adaptador una serie de funcionalidades de forma visual que ayude a convertir textos originales a otros con un lenguaje más claro y conciso.

 \setlength{\parskip}{10pt}
 
El objetivo principal de esta aplicación es la creación de un asistente para facilitar la traducción de textos a Lectura Fácil. 

 \setlength{\parskip}{10pt}


En definitiva, diseñar una página web cómoda y útil para el traductor.

\section{Estructura del documento}


Hemos seguido una serie de pasos para el desarrollo de esta memoria, la cuál se divide en los siguientes capítulos:

\begin{itemize}
	\item {\textbf{Capítulo 1}} (Introducción, motivación y estructura del documento): en este capítulo hacemos una pequeña introducción para poder entender el problema de la lectura de la cuál surge este TFG, la motivación y la estructura con explicaciones acerca de lo que vamos ha exponer en cada uno de los capítulos
	
 \setlength{\parskip}{10pt}
	
	\item{\textbf{ Capítulo 2}} (Estado del arte): en este capítulo, después de la investigación que nos ha llevado, explicamos que es la Lectura Fácil, cómo surge, a quién va dirigido, etc.

\end{itemize}

