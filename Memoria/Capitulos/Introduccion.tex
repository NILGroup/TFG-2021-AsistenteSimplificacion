\chapter{Introducción}
\label{cap:introduccion}

\chapterquote{La lectura no da al hombre sabiduría; le da conocimientos}{William Somerset Maugham}

\section{Motivación}

La lectura es comprender cualquier tipo de información de manera visual o de forma escrita. El poder leer como por ejemplo un libro no es una obligación, es un derecho que cualquier persona debe tener a su alcance. De eso modo, el lector, debe trasladar palabras, símbolos, imágenes o números dentro de la mente, y así aprender.

 \setlength{\parskip}{10pt}

 Sin embargo, hay personas con deterioros funcionales o que por algún otro motivo no pueden permitírselo, es decir, se encuentran con dificultades de comprensión lectora y comunicación. Esto es el principal objetivo que persigue la Lectura Fácil, la accesibilidad cognitiva, un sistema de lecto-escritura adaptado.
 
 \setlength{\parskip}{10pt}
 
Hoy en día vivimos en un mundo rodeado de la tecnología, teniendo a nuestra disposición dispositivos electrónicos haciéndonos la vida más fácil e incluso más rápida. Cada vez hay más iniciativas que hacen uso de ellas para poder superar esas barreras y hacer posible la lectura a todo el mundo, adaptándolas a formatos accesibles. 

 \setlength{\parskip}{10pt}
 
Puesto que el contenido en la actualidad crece a un ritmo vertiginoso, nos surge la idea del desarrollo de una aplicación, para ayudar a las personas, que hacen posible esas adaptaciones y dándoles apoyo en su día a día.


\section{Objetivos}
Esta aplicación consiste en una herramienta que ponga al servicio del adaptador una serie de funcionalidades de forma visual que ayude a convertir textos originales a otros con un lenguaje más claro y conciso.

 \setlength{\parskip}{10pt}
 
El objetivo principal de esta aplicación es la creación de un asistente para facilitar la traducción de textos a Lectura Fácil. 

 \setlength{\parskip}{10pt}


En definitiva, diseñar una página web cómoda y útil para el traductor.

\section{Estructura del documento}


Hemos seguido una serie de pasos para el desarrollo de esta memoria, la cuál se divide en los siguientes capítulos:

\begin{itemize}
	\item {\textbf{Capítulo 1}} (Introducción, motivación y estructura del documento): en este capítulo hacemos una pequeña introducción para poder entender el problema de la lectura de la cuál surge este TFG, la motivación y la estructura con explicaciones acerca de lo que vamos ha exponer en cada uno de los capítulos
	
 \setlength{\parskip}{10pt}
	
	\item{\textbf{ Capítulo 2}} (Estado del arte): en este capítulo, después de la investigación que nos ha llevado, explicamos que es la Lectura Fácil, cómo surge, a quién va dirigido, etc.

\end{itemize}

