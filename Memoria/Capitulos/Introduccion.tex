\chapter{Introducción}
\label{cap:introduccion}

\chapterquote{La lectura no da al hombre sabiduría; le da conocimientos}{William Somerset Maugham}

\section{Motivación}
Hoy en día, el ser humano tiene multitud de formas de favorecer su aprendizaje. Una de ellas es la lectura. En el mundo en el que vivimos podemos ejercerla prácticamente a través de diversos medios, bien sea mediante libros, redes sociales, televisión o prensa. Es un derecho que cualquier persona debe tener a su alcance. De ese modo, un lector puede relacionar palabras, símbolos, imágenes o números dentro de su mente, y así aprender.


 \setlength{\parskip}{10pt}

 Se dan circunstancias que hacen que este proceso de lectura no sea tan trivial, sino que se necesita una ``ayuda'' para que sea más accesible comprender la información que se transmite. Es el caso de aquellas personas que presentan algún tipo de discapacidad cognitiva, edad avanzada o no está familiarizados con el lenguaje por cualquier motivo, provocando una barrera entre ellos y la lectura.
 
  Superar esta barrera es el principal objetivo que persigue la Lectura Fácil (LF), favoreciendo la accesibilidad a estas personas por medio de textos adaptados de manera que una lectura que perciban como compleja de comprender se convierta en texto que transmita, de una manera más simple, la misma idea facilitando así la comunicación entre la lectura y el lector.
 
 \setlength{\parskip}{10pt}
 

La adaptación manual de textos o documentos a Lectura Fácil es muy costosa y lenta. Teniendo en cuenta que la aparición de nueva información (noticias, blogs, redes sociales...) está en constante crecimiento, se necesitaría mantener el mismo ritmo de adaptación a LF. Sin embargo, resulta difícil elaborar manualmente un texto que cumpla con las características de la LF. 
 

\setlength{\parskip}{10pt}
Vivimos en un mundo rodeado de avances tecnológicos, teniendo a nuestra disposición un gran número de dispositivos o herramientas. Cada vez hay más iniciativas que hacen uso de ellas para poder superar esas barreras y hacer posible la lectura a todo el mundo, adaptándolas a formatos accesibles. 


\setlength{\parskip}{10pt} 
 
Así pues, surge la idea del desarrollo de una aplicación que aúna funcionalidades que cumplan con las características y acciones de la LF, proporcionando al usuario editor una ayuda adicional, minimizando su esfuerzo cuando tiene que adaptar un texto a LF.


\section{Objetivos}
La herramienta pone al servicio del editor del texto una serie de funcionalidades y acciones de forma visual e interactiva que ayuden a convertir textos originales a otros con un lenguaje más claro y conciso y de manera rápida, además de permitirle realizar ajustes manuales en todo momento. 

 \setlength{\parskip}{10pt}
 
Así, el objetivo principal es ayudar a personas facilitando la adaptación de textos a Lectura Fácil con el uso de una aplicación interactiva que les permite hacer transformaciones sintácticas y léxicas mediante el uso de técnicas para el procesamiento del lenguaje natural. 

 

\section{Estructura del documento}


Hemos seguido una serie de pasos para el desarrollo de esta memoria, la cuál se divide en los siguientes capítulos:

\begin{itemize}
	\item {\textbf{Capítulo 1}} (Introducción, motivación y estructura del documento): en este capítulo, traducido también al inglés, hacemos una pequeña introducción para poder entender el problema de las dificultades de lectura de la cuál surge este TFG, la motivación y la estructura con explicaciones acerca de lo que vamos a exponer en cada uno de los capítulos.
	
 \setlength{\parskip}{10pt}
	
	\item{\textbf{Capítulo 2}} (Estado del arte): en este capítulo, explicamos que es la Lectura Fácil, cómo surge, a quién va dirigida, su identificación, pautas, niveles y tareas en una adaptación. También hablaremos de asociaciones, aplicaciones y materiales adaptados a Lectura Fácil. 

 \setlength{\parskip}{10pt}

\item{\textbf{Capítulo 3}} (Herramientas y tecnologías utilizadas): en este capítulo hablaremos de las tecnologías que hemos usado para el desarrollo de la aplicación.
 \setlength{\parskip}{10pt}

\item{\textbf{Capítulo 4}} (Asistente web interactivo para la simplificación de textos a Lectura Fácil): en este capítulo se describirán los requisitos necesarios que debe cumplir la aplicación web. Además, haremos un recorrido por las diferentes interfaces de la aplicación y los detalles de su uso. 

\item{\textbf{Capítulo 5}} (Implementación): en este capítulo detallaremos la arquitectura en la que está basada nuestro asistente, así como se han implementado los diferentes servicios web externos y librerías tanto de la parte del servidor como de la aplicación. 
\item{\textbf{Capítulo 6}} (Conclusiones y trabajo futuro): en este capítulo, también en inglés,  haremos unas valoraciones finales acerca del asistente y damos algunas ideas sobre ciertos desarrollos futuros que se podrían abarcar.

\item{\textbf{Capítulo 7}} (Trabajo individual): en este capítulo mostramos las aportaciones realizadas por cada uno de los miembros durante el proyecto.



\end{itemize}

